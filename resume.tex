\documentclass{article}

\usepackage{mydocument}
\usepackage{myscience}

\title{Kale Kundert}
\author{2325 Bryant Street\\San Francisco, CA 94110\\(650) 814-8603}
\date{}

\begin{document}

 \maketitle

 \pagestyle{empty}
 \thispagestyle{empty}

 %\section*{Research Goals}

 % Protein design can be succinctly defined as the search for sequences that are 
 % compatible with some desired 3D structure.  Computational design methods 
 % currently evaluate only the single lowest energy structures predicted for 
 % each sequence.  While this approach has led to a remarkable number of 
 % successful designs, the incorrect assumption that proteins adopt a single 
 % ground state has become limiting.  In reality, proteins occupy an ensemble of 
 % different conformations in their ground states.  These ground state ensembles 
 % are intimately involved in processes like allostery, binding, and even 
 % catalysis.  To incorporate phenomena like these into future designs, protein 
 % design must be redefined as the search for sequences that are compatible with 
 % some desired ensemble.

 \section*{University Education}

  \begin{description}
   \item[University of California, San Francisco] Began pursuing a doctoral 
   degree in biophysics in the Fall of 2011.  Currently working under the 
   mentorship of Tanja Kortemme.
   
   \item[University of California, Berkeley] Attended from the Fall of 2009 to
   the Spring of 2011.  Earned a 3.84 GPA and graduated with a Bachelor of 
   Science degree in the field of chemical biology.
   
   \item[Foothill Community College] Attended from the Fall of 2007 to the
   Spring of 2009.  Completed 113 quarter units and earned a 3.93 GPA.
  \end{description}

 \section*{Academic Awards}

  \begin{description}
   \item[\href{http://chemistry.berkeley.edu/commencement/address/pdf/commencement-program-2011.pdf}{Hypercube 
   Scholar Award}] Earned in the spring of 2011 upon graduation from Berkeley.  
   Awarded by the College of Chemistry to the student with the most promise in 
   the field of computational chemistry.
  \end{description}

 \section*{Research Experience}

  \begin{description}
   \item[Ensembles in Protein Design] Currently working with the Kortemme group 
   to rationally manipulate protein conformational ensembles.  The project has 
   required knowledge of statistical mechanics and the Rosetta protein design 
   software suite.
   
   \item[Light-Switchable MmCpn] Worked with the Kortemme group to develop a 
   computational protocol to predict regions within proteins where small 
   perturbations will significantly affect dynamics.  This protocol was applied 
   to the design of a light-switchable variant of the MmCpn chaperonin.  Gained 
   experience using libraries for linear algebra and normal mode analysis.

   \item[\textkappa{}-Opioid Receptor Ligands] Worked with the Shoichet group 
   to computationally screen for novel ligands of the \textkappa{}-opioid 
   receptor.  Gained experience performing ligand docking calculations and 
   running jobs on a cluster.

   \item[Site-Specific Protein Crosslinking] Worked with the Francis group to 
   build compounds composed of anilines and aminophenols that could be used to 
   make very selective cross-links between proteins.  Gained experience 
   performing organic synthesis.

   \item[Antibacterial Phage Capsids] Spent time in the Francis
   group working to create a phage with antibiotic properties.  Gained
   experience running bioconjugation reactions and building peptides using
   solid-phase techniques.
  \end{description}

 \section*{Programming Experience}

  \begin{description}
   \item[Python] Seven years and 150,000 lines of experience.  Familiar with
   matrix analysis, integration into C and C++, and PyMol scripting.

   \item[Fortran, C, C++, Java] Fluent in all these languages.  Capable of 
   choosing the appropriate language for different tasks.
   \end{description}

\end{document}
